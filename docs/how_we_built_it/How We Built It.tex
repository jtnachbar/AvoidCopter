\documentclass{article}
 
\begin{document}
\section{How we built it:}
Once our team had the idea for the quadcopter, we reached out to Margaret Fitch's dad, Simeon. Simeon is a drone hobbyist, and he has a lot of spare parts lying around, some of which he was willing to give to us. We met and walked away with heads full of wisdom and a box full of spare quadcopter parts. Now we just had to put them together. The first thing we investigated was the ardupilot. The ardupilot is an arduino with a bunch of sensors attached (accelerometer, gyroscope, compass, etc) and it is essential to controlling the quadcopter. The ardupilot is basically in charge of sending out signals to tell the propellers when and how fast to spin. It requires a little bit of tricky wiring, making sure that power, ground, and the signal pins to the motors are all connected to the back of the device. We also checked out the battery, which is a 3-cell lithium ion battery that operates at 11.1 volts. It's covered in duck tape because we had to take it apart to fix a broken cell that was out of series. If it starts to balloon out or smoke, run away really fast because it's about to release a cloud of toxic, corrosive gas (This has never happened but I thought I should mention it). Lithium polymer batteries store a lot of energy, so store them in dry, room temperature conditions and be careful about starting a fire. The ESC basically is the messenger between the ardupilot and the motors, taking in 5 volt signals from the ardupilot and sending out heavy-duty 11 volt current to the motors. The motors are three phase brushless motors, so switching any two leads will reverse the direction. It's important to have two motors clockwise and two counter clockwise for reasons outlined in the Mechanics page. We had to swap out motors after we fried one, and our frame is different. We designed and printed strut reinforcements to mitigate crash damage. We stole a transmitter (remote) and a receiver (little black box connected to ardupilot) from a previous project. A GPS came with the supplies from Simeon. We bought the cameras and the ODROID online. We soldered bullet connecters onto the motor connections because the connecters we had kept falling apart. We also added a telemetry module to communicate remotely with a computer, telling us flight diagnostics and allowing us to control the quadcopter through a computer. We broke a prop connector so we had to find a new one (prop connecters can be unscrewed by sticking something metal into the hole in the top and twisting counterclockwise.
\end{document}